% \ifx\handout\undefined
% \documentclass[professionalfonts]{beamer}
% \else
% \documentclass[handout]{beamer}
% \fi

\mode<presentation>
{
   \usetheme{CambridgeUS}
   \usecolortheme{crane}
   % Remove navigation symbols
   \setbeamertemplate{navigation symbols}{}
   % Real transparent
   % \setbeamercovered{transparent}
   %\usecolortheme{whale}
}
%\usepackage{pgfpages}
%\setbeameroption{show notes}

\usepackage[spanish,english]{babel}
%\usepackage[latin1]{inputenc}
\usepackage[utf8]{inputenc}
%\usepackage{ucs}
%\usepackage[utf8x]{inputenc}
%\usepackage{setspace}
%\usepackage{mathptmx}
\usepackage[T1]{fontenc}
\usepackage{hyperref}
\usepackage{bytefield}
\usepackage{graphicx}
\usepackage{ragged2e}
\usepackage{color}
\usepackage{verbatim} % Para usar \begin{comment} \end{comment}
\usepackage[absolute,overlay]{textpos}
\newcommand*\puttext[3]{  \begin{textblock}{10}(#1,#2) {#3}\end{textblock} }

\usepackage{minted} % Para código minted
\newminted{cpp}{fontsize=\small}

\newcommand{\ccfile}[1]{\inputminted[linenos=true,frame=lines,numbersep=8pt,framesep=10mm,fontsize=\]{cpp}{./codigo/#1}}
\newcommand{\ccfileex}[1]{\inputminted[linenos=true,frame=lines,xleftmargin=1cm,numbersep=8pt,framesep=1mm,fontsize=\footnotesize,firstline=5]{cpp}{./codigo/#1}}
\newcommand{\ccfileexlast}[2]{\inputminted[linenos=true,frame=lines,xleftmargin=1em,numbersep=8pt,framesep=1mm,fontsize=\footnotesize,firstline=5,lastline=#1]{cpp}{./codigo/#2}}
\newcommand{\ccfileexcont}[2]{\inputminted[linenos=true,frame=lines,xleftmargin=1em,numbersep=8pt,framesep=1mm,fontsize=\footnotesize,firstline=#1]{cpp}{./codigo/#2}}
\newcommand{\ccfileexrange}[3]{\inputminted[linenos=true,frame=lines,xleftmargin=1em,numbersep=8pt,framesep=1mm,fontsize=\footnotesize,firstline=#1,lastline=#2]{cpp}{./codigo/#3}}
%\newcommand{\cpplargefile}[1]{\inputminted[linenos=true,frame=lines,numbersep=8pt,framesep=10mm,fontsize=\scriptsisze]{cpp}{./codigo/#1}}

\newmintedfile{cpp}{fontsize=\small}
%Para tkiz
\usepackage{tikz}
\usepackage{pgf}
\usetikzlibrary{shapes.arrows,chains}
\tikzset{onslide/.code args={<#1>#2}{%
  \only<#1>{\pgfkeysalso{#2}} % \pgfkeysalso doesn't change the path
}}
\tikzstyle{highlight}=[red,ultra thick]


\graphicspath{{../figuras/}{./figuras/}}
%\usepackage{alltt}
%\usepackage{floatflt}


\usepackage{fancyvrb}
\usepackage{relsize}
%\usepackage{listings}

% Contenido de hightlight.sty de la versión 2.4.8
\newcommand{\hlstd}[1]{\textcolor[rgb]{0,0,0}{#1}}
\newcommand{\hlnum}[1]{\textcolor[rgb]{0.16,0.16,1}{#1}}
\newcommand{\hlesc}[1]{\textcolor[rgb]{1,0,1}{#1}}
\newcommand{\hlstr}[1]{\textcolor[rgb]{1,0,0}{#1}}
\newcommand{\hldstr}[1]{\textcolor[rgb]{0.51,0.51,0}{#1}}
\newcommand{\hlslc}[1]{\textcolor[rgb]{0.51,0.51,0.51}{\it{#1}}}
\newcommand{\hlcom}[1]{\textcolor[rgb]{0.51,0.51,0.51}{\it{#1}}}
\newcommand{\hldir}[1]{\textcolor[rgb]{0,0.51,0}{#1}}
\newcommand{\hlsym}[1]{\textcolor[rgb]{0,0,0}{#1}}
\newcommand{\hlline}[1]{\textcolor[rgb]{0.33,0.33,0.33}{#1}}
\newcommand{\hlkwa}[1]{\textcolor[rgb]{0,0,0}{\bf{#1}}}
\newcommand{\hlkwb}[1]{\textcolor[rgb]{0.51,0,0}{#1}}
\newcommand{\hlkwc}[1]{\textcolor[rgb]{0,0,0}{\bf{#1}}}
\newcommand{\hlkwd}[1]{\textcolor[rgb]{0,0,0.51}{#1}}
\definecolor{bgcolor}{rgb}{1,1,1}

\newcommand{\azul}[1]{\textcolor{blue}{#1}}
\newcommand{\rojo}[1]{\textcolor[rgb]{0.51,0,0}{#1}}

%Nuevos command que hacen falta por si se usa hightligh 3.1.2 (tomados de hightlight.sty)
\newcommand{\hlpps}[1]{\textcolor[rgb]{0.51,0.51,0}{#1}}
\newcommand{\hlppc}[1]{\textcolor[rgb]{0,0.51,0}{#1}}
\newcommand{\hlopt}[1]{\textcolor[rgb]{0,0,0}{#1}}
\newcommand{\hllin}[1]{\textcolor[rgb]{0.33,0.33,0.33}{#1}}


\makeatletter %only needed in preamble
% El primer valor fija el tamaño de punto y el segundo el espacio
% de las lineas
\newcommand\codigoNormal{\@setfontsize\Large{8pt}{10}}
\newcommand\codigoGrande{\@setfontsize\Large{10pt}{10}}
\newcommand\codigoPeq{\@setfontsize\Large{7pt}{7}}
\makeatother


\newcommand{\codigo}[2]{
    \vspace{0.2cm}
    \begin{flushright}
    \href{run:./codigo/#2}{\includegraphics[height=1cm]{gear_wheels}}
    \end{flushright}
    \vspace{-2cm}{\codigoNormal\input{./codigo/#1.tex}}
}
\newcommand{\codigazo}[2]{
    \vspace{0.2cm}
    \begin{flushright}
%    \href{run:./codigo/#2}{\includegraphics[height=1cm]{gear_wheels}}
    \end{flushright}
    \vspace{-2cm}{\codigoGrande\input{./codigo/#1.tex}}
}
\newcommand{\codigopeq}[2]{
    \vspace{0.2cm}
    \begin{flushright}
    \href{run:./codigo/#2}{\includegraphics[height=1cm]{gear_wheels}}
    \end{flushright}
    \vspace{-2cm}{\codigoPeq\input{./codigo/#1.tex}}
}
\newcommand{\trozocodigo}[1]{
    \vspace{0.2cm}
    {\codigoNormal\input{./codigo/#1.tex}}
  }
\newcommand{\trozocodigopeq}[1]{
    \vspace{0.2cm}
    {\codigoPeq\input{./codigo/#1.tex}}
}
\newcommand{\trozocodigazo}[1]{
    \vspace{0.2cm}
    {\codigoGrande\input{./codigo/#1.tex}}
}
\newcommand{\trocitocodigo}[1]{
    \vspace{0.2cm}
    {\tiny\input{./codigo/#1.tex}}
}
\newcommand{\trocitocodigazo}[1]{
    \vspace{0.2cm}
    {\small\input{./codigo/#1.tex}}
}

\newcommand{\codeinline}[1]{
    {{\small\tt #1}}
}

\AtBeginSection[] % Do nothing for \section*
{ 
\begin{frame}<beamer>
\frametitle{Contenido del tema}
%{\footnotesize
  \tableofcontents[currentsection,
        hideothersubsections]
%} 
\end{frame} 
}
\title[Fundamentos de Programación]{Fundamentos de Programación}
\date[]{}

% \defbeamertemplate*{headline}{infolines theme}
% {
%   \leavevmode%
%   \hbox{%
%     \begin{beamercolorbox}[wd=.5\paperwidth,ht=2.65ex,dp=1.5ex,right]{section in head/foot}%
%       \usebeamerfont{section in head/foot}\insertsectionhead\hspace*{2ex}
%     \end{beamercolorbox}%
%     \begin{beamercolorbox}[wd=.5\paperwidth,ht=2.65ex,dp=1.5ex,left]{subsection in head/foot}%
%       \usebeamerfont{subsection in head/foot}\hspace*{2ex}\insertsubsectionhead
%     \end{beamercolorbox}}%
%   \vskip0pt%
% }